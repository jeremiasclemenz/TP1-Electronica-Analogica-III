\section{Conclusiones}

En el trabajo practico N° 1 se realizara el estudio de un circuito de acoplamiento interetapas. Los circuitos interetapa se utilizan en sistemas 
de comunicacion para adaptar impedancia y sintonizar en una frecuencia determinada, permitiendo maxima transferencia de energia entre etapas.
En el practico construiremos el circuito resonante, armando la bobina y utilizando capacitores comerciales, donde tendremos que cumplir valores de frecuencia central, ancho 
de banda, factor de calidad e impedancia de entrada y salida.
