\section{Conclusiones}

En este trabajo se diseñó y construyó un circuito resonante, aplicando conceptos abordados en clase. El resultado de los cálculos y las mediciones no son 
exactamente iguales debido a múltiples factores.


El principal inconveniente a la hora de realizar el armado del circuito es el inductor. Para empezar, a la hora de comprar cobre para realizar el inductor,
no sabemos si el cobre es puro o no, lo que afecta a la resistencia del inductor. Por otro lado, al bobinar a mano el inductor, no tenemos la certeza de que 
la separación entre espiras sea la misma en todo el inductor, y además el cobre al tener un grosor de $2.1 \, \text{mm}$, no se puede bobinar de manera perfecta ya que 
este tiende a deformarse, lo que afecta a la inductancia del inductor. Al cambiar la inductancia, cambia la frecuencia de resonancia, lo que hace que la adaptación
de impedancias necesite otros valores de capacitores. Otro problema que tuve fue que al realizar las mediciones en dos días distintos,
los valores de las mediciones no eran iguales, lo que me llevó a tener que realizar las mediciones nuevamente. Esto se debió a que al transportar el circuito, lo modifiqué
levemente. También puede deberse a que utilicé un osciloscopio distinto, lo que puede afectar a las mediciones.


Observando la tabla de resultados, podemos ver que en $R_p$ el error es del $77.6\%$. Los cálculos realizados de la resistencia de pérdidas son de $34.3 \, \text{k}\Omega$, mientras
nosotros medimos $7.7 \, \text{k}\Omega$. Esto provocará que la potencia transferida sea menor debido a que el circuito no se encuentre adaptado. Esto se refleja en las 
mediciones de $Z_{\text{in}}$ y $Z_{\text{out}}$, donde los valores medidos no son iguales a los valores de diseño. En $Z_{\text{in}}$ el error es del $56\%$ y en $Z_{\text{out}}$ el error es del $28.5\%$.


Los capacitores utilizados son capacitores de cerámica, que cuentan con una tolerancia del $-20\%$ al $80\%$. Esto afecta a la frecuencia de resonancia y además a la adaptación
de impedancias. 


Otro problema que afecta radicalmente las mediciones son los instrumentos de medición. En este caso, aunque utilizamos conectores BNC,
se agrega la capacidad parásita de los cables, del mismo osciloscopio y del generador de señales. 



La solución al problema del trabajo artesanal del inductor es utilizar un bobinador de inductores, que nos permita bobinar inductores de manera más precisa. Utilizar calibre 
para ajustar, si es necesario, la distancia entre espiras. Esto nos permitirá tener una inductancia más precisa y una resistencia de pérdidas más cercana a la calculada.
Otra solución es utilizar inductores comerciales, que cuentan con una inductancia y resistencia de pérdidas específica. Además, para mejorar la adaptación de impedancias
podrían utilizarse capacitores de poliéster de $1\%$ de tolerancia, aunque encarecería el trabajo práctico. Y la última mejora podría ser utilizar un osciloscopio 
de alta frecuencia y un generador de señales de alta frecuencia, para disminuir la capacidad parásita de los cables y de los instrumentos de medición. Además, para medir 
la adaptación de impedancias, podría utilizarse un roímetro, que nos permitiría medir la potencia transferida y así saber si el circuito se encuentra adaptado o no.



Finalmente, considerando los problemas anteriormente mencionados y las soluciones propuestas, el trabajo nos permite tener un mayor entendimiento de los circuitos resonantes,
y además poder observar el efecto de las altas frecuencias en los circuitos.