\section{Introducción}

En el trabajo práctico N° 1 se realizará el estudio de un circuito de acoplamiento interetapas.
Los circuitos interetapa se utilizan en sistemas de comunicación para adaptar impedancia y sintonizar en una frecuencia determinada,
 permitiendo máxima transferencia de energía entre etapas. En el práctico construiremos el circuito resonante, montando la bobina y
 utilizando capacitores comerciales, donde tendremos que cumplir valores de frecuencia central, ancho de banda, factor de calidad e 
 impedancia de entrada y salida.
